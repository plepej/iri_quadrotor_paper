%last updated in April 2002 by Antje Endemann

\documentclass[runningheads]{llncs}
%In order to omit page numbers and running heads
%please use the following line instead of the first command line:
%\documentclass{llncs}.
%Furthermore change the line \pagestyle{headings} to
%\pagestyle{empty}.

\input{psfig.sty}

\begin{document}

\pagestyle{headings}
%In order to omit page numbers and running heads
%please change this line to
%\pagestyle{empty}
%and change the first command line too, see above.

\mainmatter

\title{Qaud-Copter 3D Navigation System From Simulation Towards Real Application}

\titlerunning{ISCV paper submition}

\maketitle

\begin{abstract}
An important labor in environmental protection and green economy is biomass estimation. In this work we present a system that is able to record large scale environmental data useful for biomass estimation, forest health estimation, and others. Overall, this system consists on an autonomous quadrotor which is able to fly inside a dense forest environment below treetops. We built a fully working navigation system by combining some state-of-art methods and  simulations in a forest environment are presented. The experiments have been done in the laboratory of the Institut de Robòtica i Informàtica Industrial. To test all the navigation system capabilities and without compromising flight safety, we combine both real and virtual environments thanks to an indoor localization system (i.e. Optitrack) as a pose estimator of the Quadcopter. We have created a 3D model of the laboratory to use it virtually and the quadrotor is spawned in it using the localization of the real platform. With this, we are able to include  several virtual obstacles as pillars and complex tree structures. Thanks to the precise pose estimation and the accurate reproduction of the lab, we also simulate a lidar sensor (i.e. Velodyne puck model) endowed below the platform.  By using all the data from the real and simulated scenarios, the developed navigation system is able to fly without colliding in complex environment like a forest.

\dots
\end{abstract}


\section{Introduction}
This work constitutes the initial steps of a project proposal in the framework of the H2020 european project EuRoC. This proposal was named “BORINOT: The Forest Drone”. The woking team  of the project was called MIRIAMM and was formed by members from Institut de Robòtica i Informàtica Industrial (IRI),and ASCAMM Technology center, both from Barcelona. This team was led by Joan Solà from IRI. One of the MIRIAMM team objective is to present a full autonomous navigation system in 3D environmental setup. The 3D navigation system will provide maneuverability for the MAV, the complete system setup is described in following chapters.

The document is structured as follows. In the next section a motivation is provided with the research plan and a brief description of the relevant state-of-art methods. Section 2 describes the system setup. The navigation system and its functionalities are shown in section 3. Section 4 goes over the simulated results and real experiments done at IRI Finally conclusions and future work are provided in sections 5 and 6 respectively. Additionally to references we added an appendix, detailing the system parameters used.

Main project motivation is the significant role of the forest ecosystems at international scale for environmental protection and green economy. Mobile robotics have not been very progressive in the forest field, where we see a high potential of technology usage. For example forest biomass should be considered as one of the foundations for a future bio-based economy, to preserve stable ecosystem and the biosphere. Kyoto and other international agreements value forests as a key element for the computation of CO2 absorption capabilities. This is a step forward in automation of tasks in such vast, remote and poorly accessible sites exploration, where aerial robotics appear as a reasonable solution. Currently, forest biomass estimation is done by the inspection carried by specialists on the areas of cultivation. The results give a rough estimation of volumes and diameters of trees. Also the physical inspection by the specialist is limited to the exploitation plans of large forest areas. Other CO2 absorption are estimated by airborne or satellite imaginary from altitude, which in some areas is difficult due to frequent clouds, and only observe the upper tree coverage. 

This work is focusing on building a 3D navigation system for forest exploration using MAV. The MAV have the task to explore, plan and navigate in a dense forest environment. Main objective is to build a 3D map of the forest, based on which the location of standing trees and volumes of individual trees can be assessed. We believe that automated MAV (Micro Aerial Vehicle) can contribute to the preservation of the ecosystems and can reduce the risks and impact of natural disasters such as fires.

TODO: 
- Forest motivation
- Costs motivation
- Development Framework (name for the system)

\section{State Of Art}

TODO:
- Forest exploration with MAV
- Analysis state of art for each method


\section{Development Framework Design}

TODO:
- navigation module presentation
- Fig: scheme
- Fig: virtual + reality
- 

\section{Results}

TODO:
- Results in simulation?
- Results with real setup
- Fig: Show graphs with paths and waypoints

\section{Conclusion}

TODO:
- Sum the results
- Future Work

\section{References}


\end{document}
